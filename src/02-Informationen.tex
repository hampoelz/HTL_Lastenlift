%\begin{noindent}
\begin{markdown}
# Informationen für die Projektdatenbank

## Ausgangslage

Es soll ein Lastenlift realisiert werden, damit Lasten von einem Stock in den darüber liegenden Stock befördern werden können. Es sind keine ähnlichen Projekte vorhanden, auf die zurückgegriffen werden könnten.

## Partner und Betreuungspersonen

Die Betreuungspersonen des Projektes sind Heimo Blattner und Tanzer Thomas.

## Untersuchungsanliegen der individuellen Themenstellung

Um den Lastenlift zeitgemäß fertig stellen zu können, wurde das Projekt in kleinere Arbeitsschritte unterteilt:

- Zeiteinteilung
- Zustandstabelle
- Zustandsdiagramm
- Zustandstabelle _(für Sensoren und Aktoren)_
- Anschlussschema der SPS
- Auswahl der Sensoren
- Implementieren der Steuerung _(in Automation Studio)_
- Erstellen einer Visualisierung _(in Automation Studio)_

## Zielsetzung
Das Ziel dieses Projektes ist es, einen funktionierenden Lastenlift zu bauen.

## Geplantes Ergebnis

Der Lastenlift, der in einem Gebäude installiert ist, soll Lasten von einem Stock in den nächsten Stock befördern.
\end{markdown}
%\end{noindent}